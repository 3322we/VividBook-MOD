\documentclass[
        %device=normal,
        %paper=beige,
        %color=blue,
        %chapterhead=off,
        lang=cn,
        chinesefont=founder,
        scheme=chinese,
        %font=stix2,
        %mode=simple,
        11pt
]{elegantbook}

%%% ========== 主文件导言区配置 ========== 

%% ======== 加载自定义宏包 ======== 
\usepackage{Macros/Econfig}
\usepackage{Macros/Ecodecopy}
\usepackage{Macros/Eenum}
\usepackage{Macros/Efiguretable}
\usepackage{Macros/Emath}
%% ======== 宏包加载完毕 ======== 

%% ======== 封面选项 ======== 
% ------ 标题 ------
\title{VividBook MOD}
\subtitle{VividBook Modified Version}

% ------ 写作信息 ------
\author{3322we (大史莱姆)}
\date{\today}
\version{第 $1$ 版}
\bioinfo{邮箱}{tarquedumonde@gmail.com}

% ------ 寄语 ------
\extrainfo{一套功能超多, 报错又少的漂亮模板!}

% ------ 封面图片 ------
\logo{kling.png}
\cover{thecover.png}

% ------ 修改标题页的颜色带 ------
%\definecolor{customcolor}{RGB}{245, 250, 246}
\colorlet{coverlinecolor}{second}
%% ======== 封面选项完毕 ========
%%% ========== 导言区配置完毕 ==========  %配置导言区

\begin{document}
%% ======== 封面 ======== 
\chapterimage{thetableofcontents.png}
\maketitle

    % !TeX root = ../book.tex
\frontmatter
\thispagestyle{empty}
\newpage
\begin{center}
	\textbf{\LARGE 模板概述}
\end{center}
\begin{ascolorbox1}{模板相关说明} 

\textit{VividBook MOD} 是在 \textit{雨霓同学}\& \textit{Azure1210} 开发的 \LaTeXe{} 模板 \textit{VividBooK} 的基础上修改制作而成, 继承了原 \textit{ElegantBook} 模板丰富的选项特色. 以下是本模板主要会用到的选项与值:

 \begin{itemize}
     \item \texttt{device=normal, pad}, 随阅读设备更换纸张尺寸. 但一般用不到, 平板也可用 normal.
     \item \texttt{paper=white,beige}, (新) 定义纸张颜色 (不含封面).
     \item \texttt{color=green,cyan,blue,orange,black}, 定义颜色主题. 
     \item \texttt{chapterhead=on,off}, (新) 定义章标题样式.
     \item \texttt{lang=cn,en,fr} (新增法语) 文档使用的语言.
     \item \texttt{chinesefont=ctexfont,founder,nozhfont} 中文字体. 
     \item \texttt{scheme=plain,chinese} 页眉编号数字样式, 中文下为汉字编号. 功能比较单一.
     \item \texttt{font=cm,stix2,termes,garamond,nofont} (大改) 西文字体 (正文和数学). 
     \item \texttt{mode=fancy,simple} 定理环境样式.
 \end{itemize}

 有了这些选项, 使用者可凭一套模板写出几种完全不同的风格.

 \textit{VividBook MOD} 使用的图片大部分继承自 \textit{VividBooK}.
\end{ascolorbox1}
\begin{flushright}
	大史莱姆 \\
	2026/01/16 20:30
\end{flushright}

\frontmatter

%% ======== 前言 ======== 
\frontmatter

\chapter*{前言}

    % !TeX root = ../book.tex
%% ========== 前言 ==========

本文档介绍 \textit{VividBook MOD} 模板的使用方法和注意事项. 基于自用的使用情况, 本文档的说明不能保证详尽, 如有不清晰之处, 请参考 \textit{ElegantBook} 原模板的说明.

本文档使用 11pt 方正中文字体和 Computer Modern 默认西文字体, 且应用下述样式 

\begin{table}[htbp]
    \centering
    %\caption{}
    %\label{tab:}
    \begin{tabular}{c|c}
        选项 & 值 \\ \hline
        device & normal \\ 
        paper & white \\ 
        color & blue \\ 
        lang & cn \\ 
        mode & fancy \\ 
        chapterhead & on 
    \end{tabular}
\end{table}

我们不建议随意改动纸张尺寸. 一是因为个别盒子或图案在尺寸上可能并未与其他纸张尺寸适配, 二是因为规范的纸张尺寸太多, 难以做出合适的选择, 例如出版用的 16 开纸实际上有大有小, 有 186 $\times$ 260 mm, 186 \( \times  \) 240 mm 等多种裁切尺寸. 此外, A4 纸总是打印用的最常用选择. 作者预想设计的纸张尺寸选项 \texttt{format} 由于上述原因而未做. 

作者努力优化了文档类的代码, 调整和增加了不少作者认为好的功能, 并尽可能减少非必要的屎山代码, 但由于作者水平与精力有限, 文档的写作就不那么规范了, 尤其是英文和法文. 敬请谅解.



%% ======== 目录 ======== 
\thispagestyle{fancy}
\tableofcontents

%% ======== 正文 ======== 
\mainmatter

\partsimage{thepart.png}
\parttext{这是 PART 部分的概述环节, 用来介绍这一部分的大致内容.}
\part{模板概述}

% ------ 绪论 ------
\chapterimage{SchoolofAthens.jpg}
\chapter*{绪论}
\addcontentsline{toc}{chapter}{绪论} %将标星章节引入目录

    \subfile{Cha/cha00}

% ------ 第一章 ------
\chapterimage{thechapter1.png}
\chapter{模板使用说明}
    
    \subfile{Cha/cha01}

% ------ 第二章 ------
\chapterimage{thechapter2.png}
\chapter{模板环境介绍}
    
    \subfile{Cha/cha02}

% ------ 第三章 ------
\chapterimage{thechapter3.png}
\chapter{宏包介绍}

    \subfile{Cha/cha03}

% ------ 习题 ------
\chapter*{习题参考答案与提示}
\addcontentsline{toc}{chapter}{习题参考答案与提示}

    \subfile{Cha/chaanswers}

% ------ 索引 ------
\chapter*{索引}
\addcontentsline{toc}{chapter}{索引}

    \subfile{Cha/chaindex}

\end{document}
