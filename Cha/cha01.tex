% !TeX root = ../book.tex
\documentclass[../book.tex]{subfiles}
\graphicspath{Figure/}
\begin{document}

%% ======== 导读 ======== 
本章介绍模板的基本使用方法.

\lace
%% ======== 正文 ========
\section{模板简介}

    \phantomsection
    \subsection{定位}

\textit{VividBook MOD} 是基于 \href{https://elegantlatex.org/}{Elegant\LaTeX{} 系列模板} (以下简称原版) 的一个改版 \textit{\href{https://github.com/Azure1210/VividBooK}{VividBook}} (原名 \textit{Elegantbook} 魔改版, 以下简称魔改版) 再改版而成的 \LaTeXe{} 模板. 本模板服务于数学讲义或数学笔记的编写, 模仿正式出版物的排版风格, 增加了多种实用选项, 提高了创作的自由性. 

本模板的一大特色是提供了丰富的选项 (Options). 选项是文档类一种参数, 取 (=) 不同的值 (Values) 将实现不同的功能. 下面是声明本文档类的格式.

\begin{lstlisting}
\documentclass[
            paper=beige,  %米色纸张
            color=green, %绿色主题
            lang=cn,  %支持中文
            font=termes, %TeX Gyre Termes 系列西文字体
            chinesefont=founder,  %方正字体
            11pt, 字号,只能取整数值, 建议10-12pt之间
]{elegantbook}
\end{lstlisting}

选项取默认值时可省去不写, 例如 color 的默认值是 blue. 模板概述中已经介绍了本模板的选项类型和值. 选项将大大增强模板的自定义能力.

    \subsection{模板的优化细则}

本模板在魔改版的基础上进行了诸多改动, 包括但不限于以下各条:
\begin{itemize}
    \item 删减并整理模板文件代码, 从原来的 2400 多行简化为 1800 多行. 补充了大量代码注释, 增加模板文件的可读性.
    \item 删除了大量非必需的盒子. 如各种 ascolorbox. 这些盒子可能是用来作为主要数学环境盒子的替代品, 或单纯是备用的. 
    \item 尤其是删除了需要外部 python 程序的相关盒子. 它们也不是必需的, 而且, 如果使用者未配备好相关环境, 则文件将无法编译. 本模板舍掉了这个功能, 同时也省去了一些麻烦.
    \item 删去了少量环境, 例如版本更新记录环境 change. 此种环境是开发者专用的, 对于使用者, 尤其是数学讲义编写者而言并无用途.
    \item 增加了 paper 选项和 chapterhead 选项.
    \item 重置了字体选项, 现在可以用 font 选项同时更改西文正文和数学字体. stix2 和 termes 选项的字体采用 unicode 编码, 字符全面, 无编译警告.
    \item 大幅度调整了颜色主题. 
    \item 规范了定理环境的用法, 恢复了amsthms 默认的标签引用方式, 从而避免了因使用习惯的不同而产生的大量警告. 
    \item 修正了魔改版证明环境无法使用 \texttt{\textbackslash qedhere} 命令修改证毕符号位置的缺陷.
    \item 重做了目录样式, 使其更接近严肃出版物. chapter 改用汉字计数. subsection 在正文和目录中均不再计数\footnote{可通过 myconfig 宏包中的 \textbackslash setcounter 命令调整}, 规范了无编号章节的使用.
\end{itemize}

    \subsection{编译环境}

经过粗浅的测试, 本模板可在 Win11 + TexLive 2025 + Xe\LaTeX{} 的编译环境下编译. 其他环境下的编译情况未进行测试, 因此请谨慎使用. 尤其是尽量要使用 Xe\LaTeX 编译, 而不是使用其他编译器. 虽然文档类中通过 ifTeX 逻辑做了稳健性处理, 但由于未做测试, 可用性无法保证, 而且其他编译器对字体的支持不如 Xe\LaTeX 全面稳定. 

本模板的编译速度相较魔改版可能没有明显的提升. 作者未进行仔细比较. 如有速度需求, 请尽量少编译图片, 尤其是 part 和 chapterhead.

\section{根目录管理}

根目录是当前整个 \LaTeX{} 项目所在的大文件夹. 大文件夹下面的子文件夹格式为
\begin{center}
    Subfile/Subsubfile/... /thisfile.tex\quad \% ...表示省略号
\end{center}
子文件夹中的文件在查找上级文件夹中的文件时, 引用格式为
\begin{center}
    .. /Subfile/... /thisfile.tex\quad \% ..表示返回上一级, ...表示省略号
\end{center}
任何文件的任何路径都不要带有中文字符和特殊符号, 应使用拉丁字母和半角下划线. 尽量避免空格.

作者认为拥有良好的文件管理意识对于 \LaTeX{} 文章的编写极为重要, 因此本模板的文档管理较为严格, 为章节、图片、宏包等分别单开一级子文件夹. 子文件夹层级不要太复杂, 通常只要一级就够了. 作者建议任何文章的根目录都应遵循下述或类似于下述的配置: 

\begin{lstlisting}
- Cha/ %每章单开一个子文件, 写在这个文件夹里, 而不是写在主文件里
    -- cha00 %绪论
    -- cha01 %第一章
    -- ...
    -- prefance %前言
- Figure/ %储存图片. 也可使用 figure, image等名字, 但建议统一
    -- fig.png
    -- ...
- Macros/ %宏包库, 将所有宏包和自定义功能集成在自定义宏包中.
    -- myconfig.sty %配置宏包, 用于加载基础宏包和自定义全局命令
    -- myenum.sty %列表宏包
    -- myfiguretable.sty %图表类宏包, 包含 tikz 在内的各种图表功能
    -- mymath.sty %数学类宏包
- book.tex %主文件, 不建议用 main.tex, 因为按字母排序会排到后面, 不便操作.
- config.tex %导言区配置, 这样主文件只需要把它 input 进来, 缩短代码行数.
- elegantbook.cls %模板文件
- reference.bib %参考文献
\end{lstlisting}

\section{颜色处理}

本模板的 paper 选项用于将除封面外的纸张设置为白色 (white) 或浅米色 (beige), 白色为默认模式. 这两种颜色被 faint 颜色统一包含.

颜色主题由 color 选项设置, 包含 green, cyan, blue, orange, black 五种颜色主题, 默认为 blue. 每种主题有 structurecolor, main, second, third 四种颜色. 如下:

\begin{table}[htbp]
    \centering
    %\caption{}
    %\label{tab:}
    \begin{tabular}{c|ccccc}
        颜色 & green & cyan & blue & orange & black \\ \hline
        structurecolor & {\color[RGB]{0,195,110}\rule{1cm}{1cm}} & {\color[RGB]{31,186,190}\rule{1cm}{1cm}} & {\color[RGB]{0,174,247}\rule{1cm}{1cm}} & {\color[RGB]{255,154,24}\rule{1cm}{1cm}} & {\color[RGB]{0,0,0}\rule{1cm}{1cm}} \\ 
        main & {\color[RGB]{136,180,72}\rule{1cm}{1cm}} & {\color[RGB]{14,135,140}\rule{1cm}{1cm}} & {\color[RGB]{0,169,159}\rule{1cm}{1cm}} & {\color[RGB]{215,75,25}\rule{1cm}{1cm}} & {\color[RGB]{80,80,80}\rule{1cm}{1cm}} \\ 
        second & {\color[RGB]{255,134,24}\rule{1cm}{1cm}} & {\color[RGB]{245,122,143}\rule{1cm}{1cm}} & {\color[RGB]{255,154,24}\rule{1cm}{1cm}} & {\color[RGB]{130,212,28}\rule{1cm}{1cm}} & {\color[RGB]{120,120,120}\rule{1cm}{1cm}} \\ 
        third & {\color[RGB]{92,158,86}\rule{1cm}{1cm}} & {\color[RGB]{85,130,140}\rule{1cm}{1cm}} & {\color[RGB]{60,113,183}\rule{1cm}{1cm}} & {\color[RGB]{224,144,76}\rule{1cm}{1cm}} & {\color[RGB]{0,0,0}\rule{1cm}{1cm}}
    \end{tabular}
\end{table}

作者将原版和魔改版的 gray 主题换成了橙色 orange 主题. 追求此模板的使用者想必看重的主要是此模板的颜色, 因此我们希望使主题更多彩. 但同时, 各种盒子的颜色也出于与各主题的色调相容的目的而进行了调整, 删去了许多孤立的颜色.

\section{语言和字体}

本模板的 lang 选项支持英文 (en), 中文 (cn) 和法文 (fr) 三种语言. 对法语的支持略有瑕疵. 由于西文选项下未加载 ctex 包, 即不支持中文字符, 本模板做了许多稳健性措施. 代码虽不算优雅, 但能确保中文字符不会流入西文选项中. 

如果希望支持其他语言, 例如德语, 就需要为 lang 选项添加一个空值 de. 请把下述命令添加在类似代码附近:

\begin{lstlisting}
\DeclareVoidOption{de}{\ekv{lang=de}} 
\end{lstlisting}

将各标题的名称设置为德语, 需要添加下述命令

\begin{lstlisting}
\ifdefstring{\ELEGANT@lang}{de}{
\setlength\parindent{2em}
\renewcommand{\baselinestretch}{1.3}
\renewcommand{\contentsname}{Kapitel}
\renewcommand{\figurename}{Figur}
%不一一列举, 请参考原模板中的有关代码.
%其他一些语言可能需要用到 babel 宏包.
}{\relax}
\end{lstlisting}

本模板不能保证语言的可扩展性, 因此添加新语言请务必谨慎.

西文字体的 font 选项提供了 Computer Modern 系列 (cm), STIX Two 系列 (stix2), TeX Gyre Termes 系列 (termes) 和 Garamond 系列 (garamond) 字体. 请使用 Xe\LaTeX{}编译. 如果使用者对这一搭配不满意, 可在文档类中查找 \texttt{\textbackslash setmainfont} 命令来修改, 或者使用 nofont 模式, 自己重定义西文字体. 请注意, nofont 中未引入数学字符宏包 (如 amssymb) 等, 这是考虑到使用者未必需要数学字符, 或者希望使用 unicode-math 宏包, 此宏包与 amssymb 等宏包是冲突的. 

中文字体的 chinesefont 选项提供默认字体 (ctexfont) 和方正字体 (founder). 方正字体包括下述四种字体:
\begin{itemize}
    \item {\fangsong 方正仿宋} (FZFangSong-Z02.ttf): 用于证明环境.
    \item {\heiti 方正黑体} (FZHei-B01.ttf): 用于各节和环境标题
    \item {\kaishu 方正楷体} (FZKai-Z03.ttf): 斜体, 用于章标题和解题环境, 或通过 \texttt{\textbackslash textit} 使用.
    \item 方正书宋 (FZShuSong-Z01.ttf): 正文字体.
\end{itemize}

同样, 如需自己指定字体, 则将该选项设置为 nozhfont.

请注意, Windows 系统和 \TeX 发行版均未预装方正字体, 如需使用, 请自行下载 ttf 文件, 右键选择为所有用户安装 (确保安装在路径 \texttt{C:/Windows/Fonts} 下). \faHeart{}贴心提示: 如果安装了 TeXLive, 可在终端用 \texttt{otfinfo -i} 命令查询字体文件的详细信息, 主要是查字体族名和 PostScript 名. 

文档中可以使用 \texttt{\textbackslash emph\{<文本>\}} 命令对文本进行高亮强调, 样式为 \emph{structucolor 颜色的黑体}. 此外还可以用 \texttt{\textbackslash colorbox\{<颜色>\}\{<文本>\}} 命令对文本进行\colorbox{structurecolor!20}{荧光突出}.

\section{排版架构}

    \phantomsection
    \subsection{页面布局}

本模板的尺寸参数可在文档类的下述代码中找到:

\begin{lstlisting}
    \geometry{
    a4paper,
    top=25.4mm, bottom=25.4mm,
    left=20mm, right=20mm,
    headheight=2.17cm,
    headsep=4mm,
    footskip=12mm
  }
\end{lstlisting}

A4 尺寸便于打印, 但出版通常小于这一尺寸, 例如 186 $\times$ 260 mm 或 186 $\times$ 240 mm. 可将 geometry 参数中的 a4paper 修改为
\begin{lstlisting}
    paperwidth = <size> mm, % <size> 中填入毫米尺寸
    paperheight = <size> mm,
\end{lstlisting}
修改后, 各章封面大小和大标题框的高度和会自动根据纸张高度调整, 无需担心.

正文字号为 11pt, 可在主文件的文档类声明代码中修改. 
    
    \subsection{封面}

本模板封面支持封面图定制 (\texttt{\textbackslash logo} 和 \texttt{\textbackslash cover} 命令), 以及修改颜色带 (\texttt{\textbackslash colorlet} 命令). 封面信息可在 config.tex 文件的封面选项部分修改. 另外, 作者对封面图形做了微调, 消除了图片和颜色带之间的小缝隙.

在主文件的正文区开头处插入下述代码以编译出封面:
\begin{lstlisting}
%% ======== 封面 ======== 
    \chapterimage{thecover.png}
    \maketitle
\end{lstlisting}

    \subsection{目录}

本模板在魔改版的基础上重做了目录中各章节的字体字号和计数: 章为黑体中文编号, 页码加粗, 有编号章与无编号章左对齐顶格; 节为宋体阿拉伯数字编号, 页码不加粗, 有编号节与无编号节左对齐带缩进. 小节为楷体无编号, 页码不加粗. 各级标题在目录中的格式可在模板文件的下述代码中修改:

\begin{lstlisting}
% ------ Part 目录格式 ------
\titlecontents{part}
[-2pt] 
...
...
{\titlerule*[0.5pc]{$.$}\normalsize\contentspage}
%% ======== 目录格式制作完成 ========
\end{lstlisting}

可能略有瑕疵的地方是, 章节编号等的距离控制是作者按毫米单位手动调整的, 因此可能不绝对整齐, 且可能受字体的影响.

在主文件的正文区封面部分之后插入下述代码以编译目录:
\begin{lstlisting}
    %% ======== 目录 ========
    \frontmatter
    \thispagestyle{fancy}
    \tableofcontents
\end{lstlisting}

    \subsection{章节安排}

本模板章节的布局建议按照下述格式:

\begin{lstlisting}
-| Cha/
   --> cha00.tex 
   --| cha01.tex 
       \documentclass[../book.tex]{subfiles}
       \begin{document}
            %% ======== 导读 ========
           本章导读
           \lace 导读与正文之间的小花边, 取决于你是否喜欢.
           %% ======== 正文 ========
           \section{标题}
            111
               \phantomsection %如果设置了 subsection 无编号, 则建议带上此命令
               \subsection{标题}
               111
           \section{标题}
           ...
           \section*{附1\hspace{1em}标题} %无编号节
            \addcontentsline{toc}{section}{附1\hspace{4.6mm}标题} %将无编号节引入目录
       \end{document}
    --> cha02.tex
-| book.tex
    ...
    \begin{document}
    ...
    \mainmatter
    % ------ 部分 ------ %如不想要部分, 这几行都删去
    \partsimage{thepart.png} %每部分封面
    \parttext{该部分的概述}
    \part{该部分的名字}

    % ------ 绪论 ------
    \chapterimage{SchoolofAthens.jpg} %每章封面
    \chapter*{绪论} %无编号章
    \addcontentsline{toc}{chapter}{绪论} %将无编号章引入目录

        \subfile{Cha/cha00}

    % ------ 第一章 ------
    \chapterimage{thechapter1.png}
    \chapter{模板使用说明}
    
        \subfile{Cha/cha01}

    % ------ 第二章 ------
    \chapterimage{thechapter2.png}
    \chapter{模板环境介绍}
    
        \subfile{Cha/cha02}

    % ------ 习题 ------
    \chapter*{习题参考答案与提示} %无编号章
    \addcontentsline{toc}{chapter}{习题参考答案与提示}

        \subfile{Cha/chaanswers}

    % ------ 索引 ------
    \chapter*{索引}
    \addcontentsline{toc}{chapter}{索引}

        \subfile{Cha/chaindex}

    \end{document}
\end{lstlisting}
    
    \subsection{计数器与引用}

计数器选项为 thmcnt, 默认值为 section, 它统一地将各定理环境按节编号, 如需要按章编号, 则取 thmcnt=chapter.

本模板支持 hyperref 超链接.

\section*{附1\hspace{1em}无编号章节}
\addcontentsline{toc}{section}{附1\hspace{4.8mm}无编号章节}

这是无编号 section 的演示. 


\end{document}

