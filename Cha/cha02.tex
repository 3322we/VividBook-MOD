% !TeX root = ../book.tex
\documentclass[../book.tex]{subfiles}
\graphicspath{Figure/}
\begin{document}

%% ======== 导读 ======== 

本章介绍各种环境.

\lace
%% ======== 内容 ========

\section{定理类环境}

魔改版的定理类环境已经很出色, 本模板的定理类盒子未作过大的改动, 主要是颜色主题统一化, 以及删去了一些备用盒子. 同时, 我们继承了原版的 mode 选项, 提供 fancy 和 simple 两种模式, 定理盒子只有在 fancy 模式下才会绘制. simple 模式的风格可见 englishversion 文档. 本文档展示的是 fancy 模式.

    \phantomsection
    \subsection{定义}

\begin{definition}[可积性] \label{def:int}
设 $ f(x)=\sum_{i=1}^{k} a_i \chi_{A_i}(x)$ 是 $E$ 上的\textbf{非负简单函数},中文其中
$\{A_1,A_2,\ldots,A_k\}$ 是 $E$ 上的一个可测分割,$a_1,a_2,\ldots,a_k$ 是非负实数。定义 $f$
在 $E$ 上的积分为 $\int_a^b f(x)$
\begin{equation}\label{inter}
\int_{E} f dx = \sum_{i=1}^k a_i m(A_i) \pi \alpha\beta\sigma\gamma\nu\xi\epsilon\varepsilon.
\end{equation}
一般情况下 $0 \leq \int_{E} f dx \leq \infty$。若 $\int_{E} f dx < \infty$,则称 $f$ 在 $E$
上可积。
\end{definition}

\begin{lstlisting}
\begin{definition}[可积性] \label{def:int}
设 $ f(x)=\sum\limits_{i=1}^{k} a_i \chi_{A_i}(x)$ 是 $E$ 上的\textbf{非负简单函数},中文其中
$\{A_1,A_2,\ldots,A_k\}$ 是 $E$ 上的一个可测分割,$a_1,a_2,\ldots,a_k$ 是非负实数。定义 $f$
在 $E$ 上的积分为 $\int_{a}^b f(x)$
\begin{equation}\label{inter}
\int_{E} f dx = \sum_{i=1}^k a_i m(A_i) \pi \alpha\beta\sigma\gamma\nu\xi\epsilon\varepsilon.
\end{equation}
一般情况下 $0 \leq \int_{E} f dx \leq \infty$。若 $\int_{E} f dx < \infty$,则称 $f$ 在 $E$
上可积。
\end{definition}
\end{lstlisting}

    \subsection{定理}

\begin{theorem}[Fubini 定理] \label{thm:fubi}
(1)若 $f(x,y)$ 是 $\mathcal{R}^p\times\mathcal{R}^q$ 上的非负可测函数,则对几乎处处的
$x\in \mathcal{R}^p$,$f(x,y)$ 作为 $y$ 的函数是
$\mathcal{R}^q$上的非负可测函数,$g(x)=\int_{\mathcal{R}^q}f(x,y) dy$ 是 $\mathcal{R}^p$
上的非负可测函数。并且:
\begin{equation}
\label{eq:461}
\int_{\mathcal{R}^p\times\mathcal{R}^q} f(x,y)
dxdy=\int_{\mathcal{R}^p}\left(\int_{\mathcal{R}^q}f(x,y)dy\right)dx.
\end{equation}
(2)若 $f(x,y)$ 是 $\mathcal{R}^p\times\mathcal{R}^q$ 上的可积函数,则对几乎处处的
$x\in\mathcal{R}^p$,$f(x,y)$ 作为 $y$ 的函数是 $\mathcal{R}^q$ 上的可积函数,并且
$g(x)=\int_{\mathcal{R}^q}f(x,y) dy$ 是 $\mathcal{R}^p$ 上的可积函数。而且~\ref{eq:461}
成立。
\end{theorem}

\begin{lstlisting}
\begin{theorem}[Fubini 定理] \label{thm:fubi}
(1)若 $f(x,y)$ 是 $\mathcal{R}^p\times\mathcal{R}^q$ 上的非负可测函数,则对几乎处处的
$x\in \mathcal{R}^p$,$f(x,y)$ 作为 $y$ 的函数是
$\mathcal{R}^q$上的非负可测函数,$g(x)=\int_{\mathcal{R}^q}f(x,y) dy$ 是 $\mathcal{R}^p$
上的非负可测函数。并且:
\begin{equation}
\label{eq:461}
\int_{\mathcal{R}^p\times\mathcal{R}^q} f(x,y)
dxdy=\int_{\mathcal{R}^p}\left(\int_{\mathcal{R}^q}f(x,y)dy\right)dx.
\end{equation}
(2)若 $f(x,y)$ 是 $\mathcal{R}^p\times\mathcal{R}^q$ 上的可积函数,则对几乎处处的
$x\in\mathcal{R}^p$,$f(x,y)$ 作为 $y$ 的函数是 $\mathcal{R}^q$ 上的可积函数,并且
$g(x)=\int_{\mathcal{R}^q}f(x,y) dy$ 是 $\mathcal{R}^p$ 上的可积函数。而且~\ref{eq:461}
成立。
\end{theorem}
\end{lstlisting}

    \subsection{公理}

\begin{axiom}{皮亚诺公理}
皮亚诺的这五条公理用非形式化的方法叙述如下:
\begin{enumerate}
\item 1是自然数;
\item 每一个确定的自然数a,都有一个确定的后继数$a'$,$a'$也是自然数,
(一个数的后继数就是紧接在这个数后面的数,例如,1的后继数是2,2的后继数是3等等);
\item 对于每个自然数$b,c$ $b=c$当且仅当b的后继数=c的后继数;
\item 1不是任何自然数的后继数;
\item
任意关于自然数的命题,如果证明了它对自然数1是对的,又假定它对自然数n为真时,可以证明它对n'也真,
那么,命题对所有自然数都真。(这条公理保证了数学归纳法的正确性)
\end{enumerate}
若将0也视作自然数,则公理中的1要换成0。
\end{axiom}

\begin{lstlisting}
\begin{axiom}{皮亚诺公理}
皮亚诺的这五条公理用非形式化的方法叙述如下:
\begin{enumerate}
\item 1是自然数;
\item 每一个确定的自然数a,都有一个确定的后继数$a'$,$a'$也是自然数,
(一个数的后继数就是紧接在这个数后面的数,例如,1的后继数是2,2的后继数是3等等);
\item 对于每个自然数$b,c$, $b=c$当且仅当b的后继数=c的后继数;
\item 1不是任何自然数的后继数;
\item
任意关于自然数的命题,如果证明了它对自然数1是对的,又假定它对自然数n为真时,可以证明它对n'也真,
那么,命题对所有自然数都真。(这条公理保证了数学归纳法的正确性)
\end{enumerate}
若将0也视作自然数,则公理中的1要换成0。
\end{axiom}
\end{lstlisting}

    \subsection{公设}

\begin{postulate}{皮亚诺公设}
皮亚诺的这五条公理用非形式化的方法叙述如下:
\begin{enumerate}
\item 1是自然数;
\item 每一个确定的自然数a,都有一个确定的后继数$a'$,$a'$也是自然数,
(一个数的后继数就是紧接在这个数后面的数,例如,1的后继数是2,2的后继数是3等等);
\item 对于每个自然数$b,c$, $b=c$当且仅当b的后继数=c的后继数;
\item 1不是任何自然数的后继数;
\item
任意关于自然数的命题,如果证明了它对自然数1是对的,又假定它对自然数n为真时,可以证明它对n'也真,
那么,命题对所有自然数都真。(这条公理保证了数学归纳法的正确性)
\end{enumerate}
若将0也视作自然数,则公理中的1要换成0。
\end{postulate}

\begin{lstlisting}
\begin{postulate}{皮亚诺公设}
皮亚诺的这五条公理用非形式化的方法叙述如下:
\begin{enumerate}
\item 1是自然数;
\item 每一个确定的自然数a,都有一个确定的后继数$a'$,$a'$也是自然数,
(一个数的后继数就是紧接在这个数后面的数,例如,1的后继数是2,2的后继数是3等等);
\item 对于每个自然数$b,c$, $b=c$当且仅当b的后继数=c的后继数;
\item 1不是任何自然数的后继数;
\item
任意关于自然数的命题,如果证明了它对自然数1是对的,又假定它对自然数n为真时,可以证明它对n'也真,
那么,命题对所有自然数都真。(这条公理保证了数学归纳法的正确性)
\end{enumerate}
若将0也视作自然数,则公理中的1要换成0。
\end{postulate}
\end{lstlisting}

    \subsection{引理}

\begin{lemma}[某某引理]
已知函数 $y=f[g(x)]$, 若 $u=g(x)$ 在区间 $(\mathrm{a}, \mathrm{b})$ 上是增函数, 其值域
$(\mathrm{c}, \mathrm{d})$, 又函数 $y=f(u)$ 在 $(\mathrm{c}, \mathrm{d})$ 上也 是增函数,
那么复合函数 $y=f[g(x)]$ 在 $(\mathrm{a}, \mathrm{b})$ 上是增函数。
\end{lemma}

\begin{lstlisting}
\begin{lemma}[某某引理]
已知函数 $y=f[g(x)]$, 若 $u=g(x)$ 在区间 $(\mathrm{a}, \mathrm{b})$ 上是增函数, 其值域
$(\mathrm{c}, \mathrm{d})$, 又函数 $y=f(u)$ 在 $(\mathrm{c}, \mathrm{d})$ 上也 是增函数,
那么复合函数 $y=f[g(x)]$ 在 $(\mathrm{a}, \mathrm{b})$ 上是增函数。
\end{lemma}
\end{lstlisting}

    \subsection{命题}

这个用的比定理少.

\begin{proposition}
在域 $F$ 上的线性空间 $V$ 中,设向量组 $\alpha_{1}, \cdots, \alpha_{s}$ 线性无关,则向量
$\beta$ 可以由向 量组 $\alpha_{1}, \cdots, \alpha_{s}$ 线性表出的充分必要条件是 $\alpha_{1},
\cdots, \alpha_{s}, \beta$ 线性相关。
\end{proposition}

\begin{lstlisting}
\begin{proposition}
在域 $F$ 上的线性空间 $V$ 中,设向量组 $\alpha_{1}, \cdots, \alpha_{s}$ 线性无关,则向量
$\beta$ 可以由向 量组 $\alpha_{1}, \cdots, \alpha_{s}$ 线性表出的充分必要条件是 $\alpha_{1},
\cdots, \alpha_{s}, \beta$ 线性相关。
\end{proposition}
\end{lstlisting}

    \subsection{推论}

\begin{corollary}
$n$ 元齐次线性方程组有非零解的充分必要条件是:它的系数矩阵经过初等行变换化成的阶
梯形矩阵中, 非零行的数目$r<n$
\end{corollary}

\begin{lstlisting}
\begin{corollary}
$n$ 元齐次线性方程组有非零解的充分必要条件是:它的系数矩阵经过初等行变换化成的阶
梯形矩阵中, 非零行的数目$r<n$
\end{corollary}
\end{lstlisting}

    \subsection{性质}

这个不常用.
    
\begin{property}
命题 $\mathbf{5}$ 在域 $F$ 上的线性空间 $V$ 中,设向量组 $\alpha_{1}, \cdots, \alpha_{s}$
线性无关,则向量 $\beta$ 可以由向 量组 $\alpha_{1}, \cdots, \alpha_{s}$
线性表出的充分必要条件是 $\alpha_{1}, \cdots, \alpha_{s}, \beta$ 线性相关。
\end{property}

\begin{lstlisting}
\begin{property}
命题 $\mathbf{5}$ 在域 $F$ 上的线性空间 $V$ 中,设向量组 $\alpha_{1}, \cdots, \alpha_{s}$
线性无关,则向量 $\beta$ 可以由向 量组 $\alpha_{1}, \cdots, \alpha_{s}$
线性表出的充分必要条件是 $\alpha_{1}, \cdots, \alpha_{s}, \beta$ 线性相关。
\end{property}
\end{lstlisting}

    \subsection{假设}

这个更不常用.

\begin{assumption}[类型二]
这是假设环境1
行列式的转置和原行列式的值相等$\left|\boldsymbol{A}^{\prime}\right|=|\boldsymbol{A}|$
\end{assumption}

\begin{lstlisting}
\begin{assumption}[类型二]
这是假设环境1
行列式的转置和原行列式的值相等$\left|\boldsymbol{A}^{\prime}\right|=|\boldsymbol{A}|$
\end{assumption}
\end{lstlisting}

    \subsection{例}

\begin{example}\label{example:fixed point method 2}
设$\displaystyle
x_1=1,x_2=\frac{1}{2},x_{n+1}=\frac{1}{1+x_n}$,求$\displaystyle\lim_{n\to\infty}x_n$.
\end{example}

\begin{lstlisting}
\begin{example}\label{example:fixed point method 2}
设$\displaystyle
x_1=1,x_2=\frac{1}{2},x_{n+1}=\frac{1}{1+x_n}$,求$\displaystyle\lim_{n\to\infty}x_n$.
\end{example}
\end{lstlisting}

    \subsection{证明与解}
魔改版的证明环境 (proof) 不是由 amsthm 宏包默认提供的, 而是进行了重定义. 此外还定义了解环境 (solution). 通过本模板的微调, 它们的用法与 amsthm 宏包默认提供的完全一致.

\begin{proof}
从上例(2)证明过程看出,从该例子证明中可知形如$a_1 I+a_2 C+a_3 C^3+\cdots+a_n
C^{n-1}$的矩阵一定是循环矩阵,其第一行为$a_1,a_2,\cdots,a_n$.
\end{proof}

\begin{lstlisting}
\begin{proof}
从上例(2)证明过程看出,从该例子证明中可知形如$a_1 I+a_2 C+a_3 C^3+\cdots+a_n
C^{n-1}$的矩阵一定是循环矩阵,其第一行为$a_1,a_2,\cdots,a_n$.
\end{proof}
\end{lstlisting}

\begin{solution}
    $f$ 是 $[-\pi , \pi]$ 上的偶函数.
\end{solution}

\begin{lstlisting}
\begin{solution}
    $f$ 是 $[-\pi , \pi]$ 上的偶函数.
\end{solution}
\end{lstlisting}

如果证明或解环境的段落以一个行间公式或多行公式结尾, 那么证毕符号 \qedsquare 会被挤到一个空行去. 解决方法是在最后一行公式末尾加上 \textbackslash qedhere 命令, 使证毕符号居于最后一行公式的右端.

\begin{proof}
     这是分行公式
    \begin{align*}
        s &= 1+1+2 \\
        &=2+2 \\
        &=4. \qedhere
    \end{align*}
\end{proof}

\begin{solution}
    这是分行公式
    \begin{align*}
        s &= 1+1+2 \\
        &=2+2 \\
        &=4. \qedhere
    \end{align*}
\end{solution}

作为对比:

\begin{proof}
     这是分行公式
    \begin{align*}
        s &= 1+1+2 \\
        &=2+2 \\
        &=4. 
    \end{align*}
\end{proof}

\begin{lstlisting}
\begin{proof}
     这是分行公式
    \begin{align*}
        s &= 1+1+2 \\
        &=2+2 \\
        &=4. \qedhere
    \end{align*}
\end{proof}
\begin{solution}
    这是分行公式
    \begin{align*}
        s &= 1+1+2 \\
        &=2+2 \\
        &=4. \qedhere
    \end{align*}
\end{solution}
作为对比:
\begin{proof}
     这是分行公式
    \begin{align*}
        s &= 1+1+2 \\
        &=2+2 \\
        &=4. 
    \end{align*}
\end{proof}
\end{lstlisting}

\section{习题与注释类环境}

    \phantomsection
    \subsection{习题环境}

本模板提供了每节习题和每章章末总习题两种习题模式. 每节习题的环境是重做的, 显得更大气.

\begin{exercise}
    这是本节习题. 不计入目录. 
    \begin{enumerate}
        \item 111
        \item 222
    \end{enumerate}
\end{exercise}

\begin{problemset}
    这是本章习题. 计入目录.
    \begin{enumerate}
        \item 111
        \item 222
    \end{enumerate}
\end{problemset}

\begin{lstlisting}
\begin{exercise}
    这是本节习题. 不计入目录.
    \begin{enumerate}
        \item 111
        \item 222
    \end{enumerate}
\end{exercise}

\begin{problemset}
    这是本章习题. 计入目录.
    \begin{enumerate}
        \item 111
        \item 222
    \end{enumerate}
\end{problemset}
\end{lstlisting}

与魔改版不同, 我们的习题环境没有将 enumerate 环境合并在内, 这是因为作者有时习惯在 enumerate 环境开始之前写一些假设、记号或说明.

    \subsection{注释环境}

本模板提供三类注释环境: 注解 (note)、评注(remark) 和自定义注释(anymark). 其中, anymark 环境有一个可选标题参数. 这三类环境的样式在魔改版的基础上都有所改动.

\begin{note}
    这是一个注解.
\end{note}

\begin{remark}
    这是一个评注.
\end{remark}

\begin{anymark}
    这是一个自定义注释.
\end{anymark}

\begin{anymark}[灵活的注释]
    这是一个自定义注释.
\end{anymark}

\begin{lstlisting}
\begin{note}
    这是一个注解.
\end{note}
\begin{remark}
    这是一个评注.
\end{remark}
\begin{anymark}
    这是一个自定义注释.
\end{anymark}
\begin{anymark}[灵活的注释]
    这是一个自定义注释.
\end{anymark}
\end{lstlisting}

\section{其他盒子}

本模板删去了魔改版中大量不必要的盒子, 仅保留了两个 ascolorbox 盒子.

\begin{ascolorbox1}[子标题]{标题}
劳仑衣普桑,认至将指点效则机,最你更枝。想极整月正进好志次回总般,段然取向使张规军证回,世
市总李率英茄持伴。用阶千样响领交出,器程办管据家元写,名其直金团。化达书据始价算每百青,金
低给天济办作照明,取路豆学丽适市确。如提单各样备再成农各政,设头律走克美技说没,体交才路此
在杠。响育油命转处他住有,一须通给对非交矿今该,花象更面据压来。与花断第然调,很处己队音,程
承明邮。常系单要外史按机速引也书,个此少管品务美直管战,子大标蠢主盯写族般本。农现离门亲事
以响规,局观先示从开示,动和导便命复机李,办队呆等需杯。见何细线名必子适取米制近,内信时型
系节新候节好当我,队农否志杏空适花。又我具料划每地,对算由那基高放,育天孝。派则指细流金义
月无采列,走压看计和眼提问接,作半极水红素支花。果都济素各半走,意红接器长标,等杏近乱共。层
题提万任号,信来查段格,农张雨。省着素科程建持色被什,所界走置派农难取眼,并细杆至志本。
\end{ascolorbox1}

\begin{lstlisting}
\begin{ascolorbox1}[<subtitle>]{<title>}[<options>]
environment content
\end{ascolorbox1}
\end{lstlisting}

这是 tcolorbox 手册自带框的黑白版本,支持分页。[fisubtitlefi]是选择项,[fioptionfi] 可以自动加定义。

\begin{ascolorbox2}[子标题]{标题}
劳仑衣普桑,认至将指点效则机,最你更枝。想极整月正进好志次回总般,段然取向使张规军证回,世
市总李率英茄持伴。用阶千样响领交出,器程办管据家元写,名其直金团。化达书据始价算每百青,金
低给天济办作照明,取路豆学丽适市确。如提单各样备再成农各政,设头律走克美技说没,体交才路此
在杠。响育油命转处他住有,一须通给对非交矿今该,花象更面据压来。与花断第然调,很处己队音,程
承明邮。常系单要外史按机速引也书,个此少管品务美直管战,子大标蠢主盯写族般本。农现离门亲事
以响规,局观先示从开示,动和导便命复机李,办队呆等需杯。见何细线名必子适取米制近,内信时型
系节新候节好当我,队农否志杏空适花。又我具料划每地,对算由那基高放,育天孝。派则指细流金义
月无采列,走压看计和眼提问接,作半极水红素支花。果都济素各半走,意红接器长标,等杏近乱共。层
题提万任号,信来查段格,农张雨。省着素科程建持色被什,所界走置派农难取眼,并细杆至志本。
\end{ascolorbox2}

\begin{lstlisting}
\begin{ascolorbox2}[<subtitle>]{<title>}[<color>][<option>]
environment content
\end{ascolorbox2}
\end{lstlisting}

这是移植的样式,可以通过修改[ficolorfi]来修改颜色。可在[fioptionfi]中指定框架颜色,标题颜色和字符颜色.

\end{document}
