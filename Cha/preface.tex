% !TeX root = ../book.tex
%% ========== 前言 ==========

本文档介绍 \textit{VividBook MOD} 模板的使用方法和注意事项. 基于自用的使用情况, 本文档的说明不能保证详尽, 如有不清晰之处, 请参考 \textit{ElegantBook} 原模板的说明.

本文档使用 11pt 方正中文字体和 Computer Modern 默认西文字体, 且应用下述样式 

\begin{table}[htbp]
    \centering
    %\caption{}
    %\label{tab:}
    \begin{tabular}{c|c}
        选项 & 值 \\ \hline
        device & normal \\ 
        paper & white \\ 
        color & blue \\ 
        lang & cn \\ 
        mode & fancy \\ 
        chapterhead & on 
    \end{tabular}
\end{table}

我们不建议随意改动纸张尺寸. 一是因为个别盒子或图案在尺寸上可能并未与其他纸张尺寸适配, 二是因为规范的纸张尺寸太多, 难以做出合适的选择, 例如出版用的 16 开纸实际上有大有小, 有 186 $\times$ 260 mm, 186 \( \times  \) 240 mm 等多种裁切尺寸. 此外, A4 纸总是打印用的最常用选择. 作者预想设计的纸张尺寸选项 \texttt{format} 由于上述原因而未做. 

作者努力优化了文档类的代码, 调整和增加了不少作者认为好的功能, 并尽可能减少非必要的屎山代码, 但由于作者水平与精力有限, 文档的写作就不那么规范了, 尤其是英文和法文. 敬请谅解.

